\documentclass[10pt, compress]{beamer}

\usetheme{metropolis}
\usepackage{appendixnumberbeamer}

\usepackage{tikz-dependency}
\usepackage{caption}
\usepackage{booktabs}
\usepackage{tabularx}
\usepackage{alltt}
\usepackage[scale=2]{ccicons}

\usepackage{pgfplots}
\usepgfplotslibrary{dateplot}

\usepackage{xspace}
\newcommand{\themename}{\textbf{\textsc{metropolis}}\xspace}

% commands from the paper
\newfontfamily\gtfont[Scale=1.1,Letters=SmallCaps]{Linux Libertine O}
\newcommand{\udtag}[1]{{\ll \textsc{#1}}}
\newcommand{\gtlabel}[1]{{\gtfont #1}}
\newcommand{\udlabel}[1]{{\tt #1}}
\newfontfamily\udfont[Scale=0.9,Letters=SmallCaps]{Linux Libertine O}
\newcommand{\utag}[1]{{\udfont#1}}
\newcommand{\ufeat}[1]{{\udfont#1}}
\newcommand{\tgl}[1]{{\em #1}}
\setmonofont[Scale=MatchLowercase]{DejaVu Sans Mono}

% commands from the paper


\newcommand{\myarrow}[1][-45]{%
  \mathrel{%
    \text{$
     \begin{tikzpicture}[baseline = -0.5ex]
       \node[inner sep=0pt,outer sep=0pt,rotate = #1] (a) at (0,0)  {$\xrightarrow{}$};
    \end{tikzpicture}
    $}%
  }%
}%




\title{Class 04: SemEval shared tasks}

\begin{document}

\maketitle


\begin{frame}{Introduction}


\end{frame}

%% History and context

\begin{frame}{History}

\begin{itemize}
  \item Senseval-1
  \item Senseval-2
  \item Senseval-3
  \item SemEval-2007
  \item SemEval-2010
  \item SemEval-2012
  \item SemEval-2013
  \item SemEval-2014
  \item SemEval-2015
\end{itemize}

\end{frame}


%% This year 

\section{This year}

\begin{frame}{This year}

\begin{itemize}
  \item Twelve tasks
\end{itemize}

\end{frame}

\begin{frame}{Timeline}


\end{frame}


\section{Affect and creative language in tweets}
%%
%%    Task 1: Affect in Tweets
%%    Task 2: Multilingual Emoji Prediction
%%    Task 3: Irony Detection in English Tweets
%%

\begin{frame}{Task 1: Affect in Tweets}


\end{frame}

\begin{frame}{Possible approaches}


\end{frame}

\begin{frame}{Task 2: Multilingual emoji prediction}


\end{frame}

\begin{frame}{Possible approaches}


\end{frame}

\begin{frame}{Task 3: Irony detection in English tweets}


\end{frame}

\begin{frame}{Possible approaches}

\section{Coreference}

\end{frame}

\begin{frame}{Task 4: Character identification on multiparty dialogues}


\end{frame}

\begin{frame}{Possible approaches}


\end{frame}

\begin{frame}{Task 5: Counting events and participants within highly-ambiguous data covering a very long tail}


\end{frame}

\section{Information extraction}

\begin{frame}{Possible approaches}


\end{frame}

\begin{frame}{Task 6: Parsing time normalisations}


\end{frame}

\begin{frame}{Possible approaches}


\end{frame}

\begin{frame}{Task 7: Semantic relation extraction and classification in scientific papers}


\end{frame}

\begin{frame}{Possible approaches}


\end{frame}

\begin{frame}{Task 8: Semantic extraction from cybersecurity reports (SecureNLP)}


\end{frame}

\begin{frame}{Possible approaches}


\end{frame}

\section{Lexical semantics}

\begin{frame}{Task 9: Hypernym discovery}


\end{frame}

\begin{frame}{Possible approaches}


\end{frame}

\begin{frame}{Task 10: Capturing discriminative attributes}


\end{frame}

\begin{frame}{Possible approaches}


\end{frame}

\section{Reading comprehension and reasoning}

\begin{frame}{Task 11: Machine comprehension using commonsense knowledge}


\end{frame}

\begin{frame}{Possible approaches}


\end{frame}

\begin{frame}{Task 12: Argument reasoning comprehension task}


\end{frame}

\begin{frame}{Possible approaches}


\end{frame}

\section{Practical}

\begin{frame}{Practical}


\end{frame}

\end{document}
